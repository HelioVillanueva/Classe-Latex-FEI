% arara: pdflatex
% arara: makeindex
% arara: makeglossaries
% arara: bibtex
% arara: pdflatex
% arara: pdflatex

\documentclass{fei}
\usepackage[utf8]{inputenc}
\hypersetup{pdftitle={Manual da classe fei.cls e pacotes para desenvolvimento de texto no modelo da biblioteca do Centro Universitário da FEI},
pdfauthor={Leonardo Anjoletto Ferreira e Douglas De Rizzo Meneghetti},
pdfsubject={LaTeX},
pdfkeywords={LaTeX}
}

\author{Leonardo Anjoletto Ferreira \& Douglas De Rizzo Meneghetti}
\title{Manual da classe fei.cls e pacotes para desenvolvimento de texto no modelo da biblioteca do Centro Universitário da FEI}
\subtitulo{Segundo o manual disponibilizado em 2007}

\begin{document}

\newacronym{fei}{FEI}{Fundação Educacional Inaciana}
\newacronym{abnt}{ABNT}{Associação Brasileira de Normas Técnicas}
\newacronym{abntex}{abn\TeX}{\textit{Absurd Norms for \TeX}}

\maketitle
\printglossaries
\sumario

\chapter{INTRODUÇÃO}

Para o desenvolvimento de trabalhos acadêmicos, a biblioteca do Centro Universitário da \index{FEI}\gls{fei} utiliza um modelo baseado na norma da \index{ABNT}\gls{abnt}. O modelo da \index{FEI}\gls{fei} é baseado na \index{ABNT}\gls{abnt} pois este não é em si um padrão fixo, mas uma série de opções cuja escolha fica a cargo da instituição.

Existem diversas instituições que utilizam modelos baseados na \index{ABNT}\gls{abnt} e até mesmo a classe \gls{abntex}, porém certas configurações são necessárias para que o texto se torne o mais próximo possível do modelo requisitado pela biblioteca da \index{FEI}\gls{fei}.

O único arquivo necessário para escrever o documento seguindo o modelo proposto é a classe \texttt{fei.cls} que realiza a formatação de todo o texto, iniciando pela capa, passando pelos elementos de pré-texto, textuais, referências e, finalmente, pela formatação de anexos e apêndices.

É necessário enfatizar a necessidade de instalação de determinados pacotes, listados e descritos mais abaixo, dos quais a classe \texttt{fei.cls} depende para seu funcionamento correto. Estes pacotes estão disponíveis nas distribuições do Mik\TeX (para Windows) e \TeX~Live (para Linux).

\section{A classe da \index{FEI}FEI (fei.cls)}

A escrita da classe que formata o texto foi realizada seguindo apenas o manual disponível pela biblioteca (nesta versão, o manual utilizado desde 2007).

Toda a formatação foi realizada a partir da importação e configuração de pacotes já existentes e disponíveis nas diversas distribuições de \LaTeX existentes. Durante o desenvolvimento desta classe, buscou-se utilizar o menor número de pacotes possíveis e sempre os mais comuns de serem encontrados.

Para facilitar a escrita do texto final, alguns comandos/ambientes já existentes foram modificados e novos comandos e ambientes foram adicionados. Desta forma, espera-se que o autor tenha menos trabalho com a formatação do texto do que com a escrita do mesmo.

\section{Citações utilizando o abn\TeX}

Para a formatação de citações e referências, foi importado o pacote \verb+abntex2cite+, o qual é mantido e atualizado constantemente de acordo com as normas da \index{ABNT}\gls{abnt}. Algumas modificações foram executadas para satisfazer o padrão (obsoleto) da biblioteca.

\section{Organização do Texto}

Este capítulo descreveu a ideia geral utilizada para criar um pacote que forneça os recursos necessários para desenvolver um trabalho escrito segundo o modelo da \index{ABNT}\gls{abnt} seguido pela biblioteca do Centro Universitário da \index{FEI}\gls{fei}.

O capítulo \ref{chap:classe} trata da classe \texttt{fei.cls}, explicando os comandos e ambientes modificados e criados. O capítulo \ref{chap:referencia} apresenta uma breve explicação de como o pacote \gls{abntex}~foi utilizado e seus comandos de citação que seguem os padrões utilizados no modelo da \index{FEI}\gls{fei}. O capítulo \ref{chap:indice} ensina os comandos básicos para a criação de um índice remissivo de palavras. O capítulo \ref{chap:indice} ensina como utilizar o pacote \verb+glossaries+ na indexação e utilização de abreviaturas e símbolos, assim como a impressão das suas respectivas listas. O capítulo \ref{chap:compilando} ensina a ordem em que o projeto deve ser compilado, assim como introduz o leitor à ferramenta \verb+arara+ de compilação automatizada de projetos \LaTeX. Por fim, o capítulo \ref{chap:instalacao} guia o leitor na instalação dos diversos pacotes e programas necessários para utilização de todas as funcionalidades da classe \texttt{fei.cls}.

\chapter{CLASSE FEI.CLS}\label{chap:classe}

\section{Pacotes necessários}\label{sec:pacotes}
    
    \begin{enumerate}
        \item\texttt{geometry}: utilizado para formatar as margens da folha;
        \item\texttt{fancyhdr}: utilizado na formatação do cabeçalho;
        \item\texttt{babel}: escolha de línguas (importado pacote para português e inglês);
        \item\texttt{fontenc}: codificação da fonte;
        \item\texttt{algorithm2e}: provê comandos para a escrita de algoritmos;
        \item\texttt{mathtools}: extensões para facilitar a escrita de fórmulas matemáticas (inclui o pacote \texttt{amsmath});
        \item\texttt{times}: carrega fonte Times New Roman;
        \item\texttt{graphicx}: importação e utilização de imagens;
        \item\texttt{paralist}: para gerar listas sem quebra de linha;
        \item\texttt{multirow}: permite que uma coluna ocupe várias linhas em uma tabela;
        \item\texttt{xcolor}: utilizado para alterar cores em células de tabela;
        \item\texttt{hyperref}: gera os links entre referências no PDF;
        \item\texttt{setspace}: espacejamento entre linhas;
        \item\texttt{caption}: altera a formatação de certas legendas;
        \item\texttt{tocloft}: permite melhor personalização de itens do sumário, lista de figuras e tabelas;
        \item\texttt{pdfpages}: faz a inclusão de páginas em PDF no documento final;
        \item\texttt{ifthen}: permite a utilização de condições na geração do texto;
        \item\texttt{imakeidx}: permite a criação de um índice remissivo ao fim do texto;
        \item\texttt{glossaries}: permite a criação de listas de símbolos e abreviaturas;
        \item\texttt{abntex2cite}: formata citações e referências de acordo com o padrão \index{ABNT}\gls{abnt} 6023.
    \end{enumerate}

\section{Pacotes que faltam importar}

    \begin{enumerate}
        \item\texttt{inputenc}: codificação dos arquivos \texttt{.tex} de entrada. Depende do editor que está sendo utilizado, normalmente \texttt{latin1} ou \texttt{utf8}.
    \end{enumerate}

\section{Comandos e ambientes modificados}
    
    \subsection{\textbackslash maketitle}
    
    O comando \verb+\maketitle+ foi modificado para criar uma página no formato da biblioteca. O comando utiliza o nome fornecido em \verb+\author{}+, o título em \verb+\title{}+, o subtítulo de \verb+\subtitulo{}+ juntamente com o ano corrente para gerar a capa. O local (São Bernardo do Campo) é fixo (podendo ser alterado no arquivo da classe).

    \subsection{Ambientes itemize e enumerate}
    
    Segundo o padrão da biblioteca, toda lista deve utilizar a sequência de letras. Para que não houvesse problemas de formatação, o ambiente \texttt{itemize} foi redirecionado para utilizar o \texttt{enumerate} e este passa a utilizar a letras para a sequência de items (como utilizado na seção~\ref{sec:pacotes}).
    
    \subsection{\textbackslash listofalgorithms}
    
    O pacote \texttt{algorithm2e} já importado permite que algumas configurações sejam feitas, como a formatação da lista de algoritmos. O comando foi modificado para deixar o título centralizado e em português.
    
    \subsection{Variáveis do pacote \texttt{algorithm2e}}
    
    O pacote \texttt{algorithm2e} fornece diversos comandos para a escrita de pseudo-código em diversos idiomas. O idioma importado pela fei.cls foi o português.
    
    Exemplo:
    
\begin{algorithm}
\Entrada{Vetor \(X\)}
\Saida{Vetor \(Y\)}

\ParaCada{variável \(x_i \in X\)}{

\(y_i = x_i^2\)

}

\Retorna \(Y\)

\caption{Exemplo de algoritmo usando algorithm2e em português}
\label{lst:alg}
\end{algorithm}
    
    \subsection{Outros comandos/ambientes internos}
    
    Alguns comandos como \texttt{chapter}, \texttt{abstract} e \texttt{fontsize}, que são comandos já definidos dentro do \LaTeX foram modificados para seguir as descrições do manual da biblioteca.

    Apesar destes comandos terem sido modificados, as mudanças foram feitas de forma que a utilização dos mesmos continuasse igual, assim um texto já escrito para outro modelo poderia ser apenas recompilado utilizando esta classe.

\section{Novos ambientes}

    \subsection{\textbackslash folhaderosto}
    A folha de rosto recebe um texto já definido dependendo do tipo de texto escrito (monografia, dissertação ou tese). Este texto pode ser encontrado no manual da biblioteca e deve ser colocado entre o início e o fim do ambiente. Por exemplo,
    \begin{verbatim}
\begin{folhaderosto}
Dissertação de Mestrado apresentada ao Centro Universitário
da FEI para obtenção do título de Mestre em Engenharia Elétrica, 
orientado pelo Prof. Dr. Nome do Orientador. 
\end{folhaderosto}
    \end{verbatim}

    \subsection{\textbackslash resumo}
    O ambiente \texttt{resum} funciona da mesma forma que o ambiente \texttt{abstract}, sendo a única diferença que o \texttt{abstract} possui o comando \verb+\selectlanguage{english}+ no início e o \texttt{resumo} utiliza \verb+\selectlanguage{brazil}+.

    \subsection{\textbackslash agradecimentos}
    O ambiente de agradecimentos não possui nenhuma propriedade especial, somente centraliza o título e deixa o texto que se encontra entre seu \texttt{begin} e \texttt{end} na formatação esperada.

\section{Novos comandos}
    
    \subsection{\textbackslash subtitulo\{\}}
    Uma vez que as normas da biblioteca demandam formatações específicas para o título e subtítulo do documento (título em letras maiúsculas na capa, seguido do subtítulo em letras normais, separados por \aspas{:}), foi criado o comando \verb+\subtitulo{}+, o qual recebe o texto referente ao subtítulo do texto. Este comando pode ser usado, preferencialmente, após o comando \verb+\title{}+ no preâmbulo do documento. Título e subtítulo também aparecem na folha de rosto.
    
    \subsection{\textbackslash sumario,\textbackslash figuras,\textbackslash tabelas}
    O \LaTeX já possui comandos que criam sumário, lista de figuras e lista de tabelas, porém, para seguir o modelo necessário e facilitar a manutenção do mesmo foram criados novos comandos que geram estas listas.

    Neste caso, \verb+\sumario+ substitui \verb+\tableofcontents+, \verb+\figuras+ substitui \verb+\listoffigures+ e \verb+\tabelas+ o \verb+\listoftables+.

    \subsection{\textbackslash folhadeaprovacao\{\}}
    O comando para a folha de aprovação pode receber o argumento \texttt{ata.pdf}. Se este argumento foi passado, o comando tenta importar uma página em PDF com o mesmo nome para utilizar como folha de aprovação. Se o argumento não for passado ou for diferente de \texttt{ata.pdf}, será inserido um texto no lugar da página para marcar a posição da folha de aprovação.

    As possibilidades de utilização do comando são:
    \begin{itemize}
        \item \verb+\folhadeaprovacao{}+: insere uma página com o texto, como descrito;
        \item \verb+\folhadeaprovacao{ata.pdf}+: insere o PDF com a ata da banca.
    \end{itemize}
    
    
    \subsection{\textbackslash fichacatalografica\{\}}
    Este comando segue o mesmo princípio do \verb+\folhadeaprovacao{}+, mas espera que o argumento passado seja \texttt{ficha.pdf}. As possibilidades de utilização do comando são:
    \begin{itemize}
        \item \verb+\fichacatalografica{}+: insere uma página com o texto, como descrito;
        \item \verb+\fichacatalografica{ficha.pdf}+: insere o PDF com a ficha catalográfica fornecida pela biblioteca.
    \end{itemize}
 
 Estes dois comandos são os únicos que dependem do pacote \texttt{pdfpages} (para importar a página em PDF).

    \subsection{\textbackslash dedicatoria\{\}}
    O comando \verb+\dedicatoria{}+ recebe um argumento com a dedicatória desejada e o insere na posição especificada pelo manual da biblioteca. Por exemplo: \\ \verb+\dedicatoria{A quem eu quero dedicar o texto}+.
    
    \subsection{\textbackslash epigrafe\{\}\{\}}
    A epigrafe possui um formato especial, da mesma forma que a dedicatória. Este comando recebe dois parâmetros, sendo o primeiro a epigrafe e o segundo o autor da mesma. Por exemplo, \verb+\epigrafe{Haw-Haw!}{Nelson Muntz}+
    
    \subsection{\textbackslash aspas\{\}}
    As aspas no \LaTeX são geradas de forma diferente dos outros editores de texto e pode ser encontrada em qualquer manual sobre \LaTeX. Apenas para facilitar a inserção de aspas no formato do \LaTeX, foi criado o comando \verb+\aspas{}+ que recebe o texto desejado e o coloca entre aspas.

    Exemplo: \verb+\aspas{Texto entre aspas}+ $\to$ \aspas{Texto entre aspas}
    
    \subsection{\textbackslash marca\{\}}
    É comum precisar que certas células de uma tabela precisem ser destacadas das demais, como em cronogramas, por exemplo. O comando \verb+\marca{}+ foi feito para que a célula de uma tabela ficasse com o fundo cinza. Este é o único comando que utiliza o pacote \texttt{xcolor}.

    Exemplo:
    \begin{verbatim}
    \begin{table}[ht]
        \begin{center}
        \begin{tabular}{|c|c|c|}
        \hline
        1 & 2 & 3 \\
        \hline
        \marca{} & & \marca{} \\
        \hline
        a & b & c \\
        \hline
        \end{tabular}
        \end{center}
    \end{table}
    \end{verbatim}
    Resultado: 
    \begin{table}[ht]
        \begin{center}
        \begin{tabular}{|c|c|c|}
        \hline
        1 & 2 & 3 \\
        \hline
        \marca{} & & \marca{} \\
        \hline
        a & b & c \\
        \hline
        \end{tabular}
        \end{center}
    \end{table}
        
    \subsection{\textbackslash palavraschave\{\} e \textbackslash keyword\{\}}
    Segundo o modelo da biblioteca da \index{FEI}\gls{fei}, o resumo e o abstract devem receber no máximo 3 palavras chave. Estes comandos devem ser utilizados dentro dos respectivos ambientes e as palavras devem ser passadas como argumentos.

    Exemplo:
    \begin{verbatim}
    \begin{resumo}
    Aqui deve ser escrito o resumo do trabalho.

    \palavraschave{Resumo. Modelo da FEI. Latex}
    \end{resumo}
    \end{verbatim}
    
    \subsection{\textbackslash apendice\{\}, \textbackslash apendices\{\}, \textbackslash anexo\{\} e \textbackslash anexos\{\}}
    Para os apêndices e anexos foram criados dois comandos separados.

    O modelo da \index{FEI}\gls{fei} requer que os apêndices e anexos sejam numerados apenas quando existe mais de um no trabalho. Caso exista apenas um anexo ou apêndice, este não leva número sequencial.

    Exemplos: \\
    \verb+\apendice{Único apêndice do trabalho}+ \\
    \verb+\anexo{Único anexo do trabalho}+ \\

    \noindent{}
    \verb+\apendices{Primeiro apêndice}+\\
    \verb+\apendices{Segundo apêndice}+\\

    \noindent{}
    \verb+\anexos{Primeiro anexo}+\\
    \verb+\anexos{Segundo anexo}+\\

    \subsection{\textbackslash bibliografia\{\}}
    A utilização de referências bibliográficas a partir do \texttt{bibtex} depende do comando \verb+\bibliography{}+ que recebe o caminho até o arquivo \texttt{.bib} utilizado. Porém, a adição da página de referências ao sumário e a formatação do título da mesma dependem de outras variáveis que precisam ser definidas durante a produção do texto (o pacote \texttt{babel} substitui o nome da página de referências e este só pode ser mudado após o início do texto).

    Para facilitar, foi criado o comando \verb+\bibliografia{}+ que recebe como parâmetro o caminho para o arquivo \texttt{.bib}. Este comando realiza a formatação necessária e repassa o caminho para o comando \verb+\bibliography{}+ padrão do \LaTeX.

    Exemplo: \verb+\bibliografia{minha_bibliografia.bib}+

\chapter{REFERÊNCIA USANDO O abn\TeX}\label{chap:referencia}

    \section{O que é o abn\TeX~e como foi utilizado}

    O \gls{abntex}~(\url{https://code.google.com/p/ABNTex2/}) é um conjunto de macros (comandos e ambientes) que busca seguir as normas da \index{ABNT}\gls{abnt} para formatos acadêmicos. O pacote completo do \gls{abntex}~fornece tanto uma classe para a formatação do texto quanto um pacote para a formatação das referências bibliográficas.

    Entretanto, a \index{ABNT}\gls{abnt} fornece certas opções para que o texto seja produzido, sendo que a biblioteca do Centro Universitário da \index{FEI}\gls{fei} ficou a cargo de escolher estas formatações para seus trabalhos.

    Para a formatação correta das citações e referências de acordo com o padrão da biblioteca da \index{FEI}\gls{fei}, foi importado o pacote \verb+\index{ABNT}\gls{abnt}ex2cite-alf+, que utiliza o modelo autor-data.

    As seções a seguir disponibilizam exemplos dos comandos mais comuns. Para uma lista detalhada, o leitor é referenciado ao manual do \verb+\index{ABNT}\gls{abnt}ex2cite-alf+.

    \section{Citação no final de linha}
    A citação no final de linha deve deixar os nomes dos autores, seguido do ano, entre parenteses e em letras maiúsculas. Este resultado pode ser obtido utilizando o comando \verb+\cite{obra}+.

    Exemplo: Este texto deveria ser uma referência \verb+\cite{j:turing50}+. $\to$ Este texto deveria ser uma referência \cite{j:turing50}.

    \section{Citação durante o texto}
    Para que a citação seja feita durante o texto, o nome do autor é formatado somente com as iniciais maiúsculas e o ano entre parenteses. O pacote da \gls{abntex}~fornece o comando \verb+\citeonline{obra}+ para este caso.

    Exemplo: Segundo \verb+\citeonline{haykin99a}+, este texto deveria ser uma referência. $\to$ Segundo \citeonline{haykin99a}, este texto deveria ser uma referência.
	
	\section{Citação indireta}
	Quando se deseja citar uma obra a qual o autor não possui acesso direto a ela, pode-se citar uma outra obra que, por sua vez, cita a primeira. O \gls{abntex}~disponibiliza esse tipo de citação através do comando \verb+\apud{obra_inacessivel}{obra_acessivel}+.
	
	Exemplo: \verb+\apud{Mcc43}{RusselNo10}+ formata a citação de forma semelhante a \apud{Mcc43}{RusselNo10}.
	
	\section{Citação no rodapé}
	
	Citações no rodapé\footciteref{j:turing50} são feitas usando o comando \verb+\footciteref{obra}+.
	
	\section{Citações múltiplas}
	
	Citações múltiplas podem ser utilizadas utilizando os comandos \verb+\cite{obra_1,...,obra_n}+ e \verb+\citeonline{obra_1,...,obra_m}+.
	
	Exemplos: 
	
	\verb+\cite{Mcc43,RusselNo10,haykin99a}+ \(\to\) \cite{Mcc43,RusselNo10,haykin99a}.
	
	\verb+\citeonline{Mcc43,RusselNo10,haykin99a}+ \(\to\) \citeonline{Mcc43,RusselNo10,haykin99a}.
	
	\section{Citações de campos específicos}
	
	Para citar o nome do autor em linha, utilize o comando \verb+\citeauthoronline{obra}+.

	\verb+\citeauthoronline{galilei_dialogue_1953}+ \(\to\) \citeauthoronline{galilei_dialogue_1953}
	
	Para citar o nome do autor em letras maiúsculas, utilize\verb+\citeauthor{obra}+.

	\verb+\citeauthor{galilei_dialogue_1953}+ \(\to\) \citeauthor{galilei_dialogue_1953}

	Para citar o ano de uma obra, utilize \verb+\citeyear{obra}+.
	
	\verb+\citeyear{galilei_dialogue_1953}+ \(\to\) \citeyear{galilei_dialogue_1953}

	\section{Outros exemplos}

	\verb+\Idem[p.~2]{j:turing50}+ \(\to\) \Idem[p.~2]{j:turing50}

	\verb+\Ibidem[p.~2]{j:turing50}+ \(\to\) \Ibidem[p.~2]{j:turing50}

	\verb+\opcit[p.~2]{j:turing50}+ \(\to\) \opcit[p.~2]{j:turing50}

	\verb+\passim{j:turing50}+ \(\to\) \passim{j:turing50}

	\verb+\loccit{j:turing50}+ \(\to\) \loccit{j:turing50}

	\verb+\cfcite[p.~2]{j:turing50}+ \(\to\) \cfcite[p.~2]{j:turing50}

	\verb+\etseq[p.~2]{j:turing50}+ \(\to\) \etseq[p.~2]{j:turing50}

	\chapter{ÍNDICES USANDO MAKEINDEX E XINDY}\label{chap:indice}
	
	Para a criação de índices remissivos, foi importado o pacote \texttt{imakeidx}. Este pacote utiliza o programa \texttt{xindy} na criação de arquivos auxiliares que indexam as palavras a serem colocadas no índice. Portanto, é necessário ter o \texttt{xindy} instalado para que a compilação tenha sucesso.
	
	\section{Indexando palavras}
	
	Para que uma palavra apareça posteriormente no índice, é necessário indexá-la. Para isso, usa-se o comando \verb+\index{palavra}+, o qual inclui \aspas{palavras} no arquivo auxiliar de indexação.

	Exemplo: [\ldots] a biblioteca do Centro Universitário da \verb+\index{FEI}+\index{FEI}\gls{fei} utiliza um modelo baseado na norma da \verb+\index{ABNT}+\index{ABNT}\gls{abnt} [\ldots]
	
	É possível indexar uma palavra mais de uma vez, para que todas as páginas nas quais esta palavra apareceu apareçam no índice.
	
	\section{Imprimindo o índice}

	A impressão do índice é feita utilizando o comando \verb+\indice+, o qual, além de imprimir o índice, inclui uma entrada para o mesmo no sumário.

	\chapter{LISTAS DE SÍMBOLOS E ABREVIATURAS} \label{chap:listas}
	
	Para a criação das listas de símbolos e abreviaturas, foi utilizado o pacote \verb+glossaries+, responsável por indexar termos de diferentes categorias e gerar listas destes termos. Ao contrário do índice, que indexa as palavras no decorrer do texto, o pacote \verb+glossaries+ exige que os termos sejam declarados antes de serem referenciados durante o texto. Uma boa prática para organizar tais termos consiste em declará-los ao início do documento, ou em um documento separado, o qual pode ser chamado utilizando os comandos \verb+\input+ ou \verb+\include+. Estas opções ficam a cargo do leitor. As próximas duas seções ensinarão os comandos básicos para indexação de símbolos e abreviaturas.
	
	\section{Indexando abreviaturas}
	
	A indexação de abreviaturas é feita utilizando o comando
	
	\verb+\newacronym[longplural=1]{2}{3}{4}+, onde:
	
	\begin{enumerate}
	\item 1: o significado a abreviatura no plural, escrito por extenso (\emph{opcional})
	\item 2: código que será utilizado para referenciar a abreviatura no decorrer do texto
	\item 3: a abreviatura em si
	\item 4: o significado a abreviatura, escrito por extenso
	\end{enumerate}
	
	Exemplo: \verb+\newacronym[longplural=Associações+
	
			 \verb+Brasileiras de Normas Técnicas]+
			 
			 \verb+{abnt}{ABNT}{Associação Brasileira de Normas Técnicas}+
			 
	\section{Indexando símbolos}
	
	A indexação de símbolos é feita utilizando o comando
	
	\verb+\newglossaryentry{1}{type=simbolos,+
	
	\verb+name={2},sort=3,description={4}}+, onde:
	
	\begin{enumerate}
	\item 1: código que será utilizado para referenciar a abreviatura no decorrer do texto)
	\item 2: o símbolo; caso a notação matemática seja necessária, use \verb+\ensuremath{2}+ (Cf. exemplo abaixo)
	\item 3: uma sequência de caracteres para indicar a ordenação alfabética do símbolo na lista
	\item 4: a descrição do símbolo, que aparecerá na lista
	\end{enumerate}
	
	Exemplo: \verb+\newglossaryentry{pi}{type=simbolos,+
	
			 \verb+name={\ensuremath{\pi}},sort=p,+
			 
			 \verb+description={número irracional que representa [...]}}+

	\section{Utilizando abreviaturas e símbolos indexados}
	
	O pacote \verb+glossaries+ disponibiliza os seguintes comandos para chamar os itens indexados durante o texto:
	
	\begin{enumerate}
	\item \verb+\gls{codigo}+: imprime a entrada em letras minúsculas;
	\item \verb+\Gls{codigo}+: imprime a entrada em letras maiúsculas;
	\item \verb+\glspl{codigo}+: imprime a entrada no plural;
	\item \verb+\Glspl{codigo}+: imprime a entrada no plural e em letras maiúsculas;
	\end{enumerate}
	
	Repare que, no caso das siglas, quando estas são usadas pela primeira vez, são impressas a definição seguida da sigla entre parênteses. Nas demais vezes, a sigla aparecerá sozinha. É importante ressaltar que o pacote \texttt{glossaries} adiciona às listas somente os termos que foram utilizados durante o texto. Para que todos os termos declarados apareçam, basta usar o comando \verb+\glsaddall+ no corpo do texto.
	
	\section{Imprimindo as listas}
	
	O comando \verb+\printglossaries+ imprime ambas as listas em sequência.

	\chapter{COMPILANDO O PROJETO} \label{chap:compilando}
	
	Para a utilização de todos os recursos que a classe \verb+fei.cls+ disponibiliza, é necessário compilar o projeto utilizando diversas ferramentas diferentes em ordem específica:
	
	\[\operatorname*{\text{PDF\LaTeX}}_a \to \operatorname*{\text{Bib\TeX}}_b \to \operatorname*{\text{MakeIndex}}_c \to \operatorname*{\text{MakeGlossaries}}_d \to \operatorname*{\text{PDF\LaTeX}}_e \to \operatorname*{\text{PDF\LaTeX}}_f\]
	
	onde:
	
	\begin{enumerate}
	\item gera o PDF e arquivos auxiliares básicos;
	\item lê os arquivos auxiliares criados em 1, gerando a bibliografia apenas com as referências utilizadas do arquivo \verb+.bib+ utilizado. Necessário apenas se citações e referências forem usadas no texto;
	\item lê os arquivos auxiliares, criando um ou mais arquivos de índice. Necessário apenas se houve indexação de palavras para serem adicionadas ao índice;
	\item lê os arquivos auxiliares, criando um ou mais arquivos de listas. Necessário apenas se houve indexação e utilização de símbolos e abreviaturas no texto;
	\item atualiza todas as referências através do texto, utilizando os arquivos gerados em \emph{b}, \emph{c} e \emph{d} (\emph{desnecessário se os passos b--d não foram realizados});
	\item gera o PDF final (\emph{desnecessário se os passos b--e não foram realizados}).
	\end{enumerate}
	
	\section{Compilação utilizando arara}
	
	O \texttt{arara} (\url{cereda.github.io/arara/}) é uma ferramenta de automação de compilação de projetos em \LaTeX. Seu trabalho consiste em compilar um arquivo \texttt{.tex} utilizando como instruções comentários colocados nas primeiras linhas do arquivo. Esta abordagem permite a abstração por parte do usuário de chamadas complexas a programas como \texttt{makeindex}, \texttt{xindy} e \texttt{glossaries}. Abaixo, um exemplo de como os comandos para o processo de compilação exemplificado acima ficariam em \texttt{arara}:
	
	\begin{verbatim}
% arara: pdflatex
% arara: makeindex
% arara: makeglossaries
% arara: bibtex
% arara: pdflatex
% arara: pdflatex

\documentclass{fei}
[...]
	\end{verbatim}
	
	E a chamada ao \texttt{arara} seria feita da seguinte forma:
	
	\texttt{arara manual.tex}
	
	A escolha de como compilar o projeto fica a cargo do leitor.	
	
	\chapter{INSTALAÇÃO DOS PACOTES E PROGRAMAS}	\label{chap:instalacao}
	
	Este capítulo guia o leitor na instalação dos diferentes pacotes e programas necessários para utilizar todas as funcionalidades da classe \texttt{fei.cls}.
	
	\section{Ubuntu}
	
	No Ubuntu, é necessária a instalação dos seguintes pacotes através do \texttt{apt-get}:
	
	\begin{enumerate}
	\item \texttt{texlive}: pacotes essenciais do \TeX~Live;
	\item \texttt{texlive-science}: instala pacotes científicos, como \texttt{algorithm2e} e \texttt{mathtools};
	\item \texttt{texlive-lang-portuguese}: idioma português do \texttt{babel};
	\item \texttt{texlive-publishers}: pacote \texttt{abntex2cite} da \gls{abntex}.
	\item \texttt{texlive-extra-utils}: \texttt{arara};
	\item \texttt{xindy}: criação de índices e listas usando \texttt{xindy}.
	\end{enumerate}
	
	\bibliografia{referencias}

	\indice
	
\end{document}