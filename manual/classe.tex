\chapter{CLASSE FEI.CLS}\label{chap:classe}

\section{Pacotes necessários}\label{sec:pacotes}
    
    \begin{enumerate}
        \item\verb+geometry+: utilizado para formatar as margens da folha;
        \item\verb+fancyhdr+: utilizado na formatação do cabeçalho;
        \item\verb+babel+: escolha de línguas (importado pacote para português e inglês);
        \item\verb+fontenc+: codificação da fonte;
        \item\verb+algorithmic+: provê comandos para a escrita de algoritmos;
        \item\verb+algorithm+: utilizado junto ao \texttt{algorithmic}, faz com que os algoritmos possam ser utilizados como \outralingua{float} do \LaTeX;
        \item\verb+amsmath+: extensões da \outralingua{American Mathematical Society} para facilitar a escrita de fórmulas matemáticas;
        \item\verb+times+: carrega fonte Times New Roman;
        \item\verb+graphicx+: importação e utilização de imagens;
        \item\verb+paralist+: para gerar listas sem quebra de linha;
        \item\verb+multirow+: permite que uma coluna ocupe várias linhas em uma tabela;
        \item\verb+xcolor+: utilizado para alterar cores em células de tabela;
        \item\verb+hyperref+: gera os links entre referências no PDF;
        \item\verb+setspace+: espacejamento entre linhas;
        \item\verb+caption+: altera a formatação de certas legendas;
        \item\verb+tocloft+: permite melhor personalização de itens do sumário, lista de figuras e tabelas;
        \item\verb+pdfpages+: faz a inclusão de páginas em PDF no documento final;
        \item\verb+ifthen+: permite a utilização de condições na geração do texto;
    \end{enumerate}

\section{Pacotes que faltam importar}

    \begin{enumerate}
        \item\verb+inputenc+: codificação de entrada do texto. Depende do editor que está sendo utilizado, normalmente \texttt{latin1} ou \texttt{utf8}.
        \item Referências: o pacote de referência pode ser escolhido pelo autor. O que será descrito neste texto é uma versão modificada do ABNTex.
    \end{enumerate}

\section{Comandos e ambientes modificados}
    
    \subsection{\textbackslash maketitle\{\}}
    
    O comando \verb+\maketitle{}+ foi modificado para criar uma página no formato da biblioteca. O comando utiliza o nome fornecido em \verb+\author{}+, o título em \verb+\title{}+ juntamente com o ano corrente para gerar a capa. O local (São Bernardo do Campo) é fixo (podendo ser alterado no arquivo da classe).

    O comando \verb+\maketitle{}+ pode receber um argumento com o subtítulo do trabalho seguindo um caracter de \aspas{:}. Por exemplo, neste manual foi utilizado o comando: \\ \verb+\maketitle{: Segundo o manual disponibilisado em 2007}+

    \subsection{Ambientes itemize e enumerate}
    
    Segundo o padrão da biblioteca, toda lista deve utilizar a sequência de letras. Para que não houvesse problemas de formatação, o ambiente \verb+itemize+ foi redirecionado para utilizar o \verb+enumerate+ e este passa a utilizar a letras para a sequência de items (como utilizado na seção~\ref{sec:pacotes}).
    
    \subsection{\textbackslash listofalgorithms\{\}}
    
    O pacote \verb+algorithm+ já importado permite que algumas configurações sejam feitas, como a formatação da lista de algoritmos. O comando foi modificado para deixar o título centralizado e em português.
    
    \subsection{Variáveis do pacote \texttt{algorithmic}}
    
    O pacote \verb+algorithmic+ fornece diversos comandos para a escrita de pseudo código, porém estes comandos estão em inglês. As variáveis foram traduzidas para um comando próximo dos utilizados em português.
    
    \subsection{Outros comandos/ambientes internos}
    
    Alguns comandos como \verb+chapter+, \verb+abstract+ e \verb+fontsize+  que são comandos utilizados, mas já definidos dentro do \LaTeX foram modificados para seguir as descrições do manual da biblioteca.

    Apesar destes comandos terem sido modificados, estas mudanças foram feitas de forma que a utilização dos mesmos continuasse a mesma, assim um texto já escrito para outro modelo poderia ser apenas recompilado utilizado esta classe. A única mudança é no caso do \verb+\maketitle{}+, que passa a receber um parâmetro opcional a mais.

\section{Novos ambientes}

    \subsection{\textbackslash folhaderosto\{\}}
    A folha de rosto recebe um texto já definido dependendo do tipo de texto escrito (monografia, dissertação ou tese). Este texto pode ser encontrado no manual da biblioteca e deve ser colocado entre o início e o fim do ambiente. Por exemplo,
    \begin{verbatim}
\begin{folhaderosto}
Dissertação de Mestrado apresentada ao Centro Universitário
da FEI para obtenção do título de Mestre em Engenharia Elétrica, 
orientado pelo Prof. Dr. Nome do Orientador. 
\end{folhaderosto}
    \end{verbatim}

    \subsection{\textbackslash resumo\{\}}
    O ambiente \verb+resumo+ funciona da mesma forma que o ambiente \verb+abstract+, sendo a única diferença que o \verb+abstract+ possui o comando \verb+\selectlanguage{english}+ no início e o \verb+resumo+ utiliza \verb+\selectlanguage{brazil}+.

    \subsection{\textbackslash agradecimentos\{\}}
    O ambiente de agradecimentos não possui nenhuma propriedade especial, somente centraliza o título e deixa o texto que se encontra entre seu \verb+begin+ e \verb+end+ na formatação esperada.

\section{Novos comandos}
    
    \subsection{\textbackslash sumario\{\},\textbackslash figuras\{\},\textbackslash tabelas\{\}}
    O \LaTeX já possui comandos que criam sumário, lista de figuras e lista de tabelas, porém, para seguir o modelo necessário e facilitar a manutenção do mesmo foram criados novos comandos que geram estas listas.

    Neste caso, \verb+\sumario{}+ substitui \verb+\tableofcontents+, \verb+\figuras{}+ substitui \verb+\listoffigures+ e \verb+\tabelas{}+ o \verb+\listoftables+.

    \subsection{\textbackslash folhadeaprovacao\{\}}
    O comando para a folha de aprovação pode receber o argumento \texttt{ata.pdf}. Se este argumento foi passado, o comando tenta importar uma página em PDF com o mesmo nome para utilizar como folha de aprovação. Se o argumento não for passado ou for diferente de \texttt{ata.pdf}, será inserido um texto no lugar da página para marcar a posição da folha de aprovação.

    As possibilidades de utilização do comando são:
    \begin{itemize}
        \item \verb+\folhadeaprovacao{}+: insere uma página com o texto, como descrito;
        \item \verb+\folhadeaprovacao{ata.pdf}+: insere o PDF com a ata da banca.
    \end{itemize}
    
    
    \subsection{\textbackslash fichacatalografica\{\}}
    Este comando segue o mesmo princípio do \verb+\folhadeaprovacao{}+, mas espera que o argumento passado seja \texttt{ficha.pdf}. As possibilidades de utilização do comando são:
    \begin{itemize}
        \item \verb+\fichacatalografica{}+: insere uma página com o texto, como descrito;
        \item \verb+\fichacatalografica{ficha.pdf}+: insere o PDF com a ficha catalográfica fornecida pela biblioteca.
    \end{itemize}
 
 Estes dois comandos são os únicos que dependem dos pacotes \verb+pdfpages+ (para importar a página em PDF) e \verb+ifthen+ (para verificar o argumento passado).

    \subsection{\textbackslash dedicatoria\{\}}
    O comando \verb+\dedicatoria{}+ recebe um argumento com a dedicatória desejada e o insere na posição especificada pelo manual da biblioteca. Por exemplo: \\ \verb+\dedicatoria{A quem eu quero dedicar o texto}+.
    
    \subsection{\textbackslash epigrafe\{\}\{\}}
    A epigrafe possui um formato especial, da mesma forma que a dedicatória. Este comando recebe dois parâmetros, sendo o primeiro a epigrafe e o segundo o autor da mesma. Por exemplo, \verb+\epigrafe{Haw-Haw!}{Nelson Muntz}+
    
    \subsection{\textbackslash outralingua\{\}}
    Por ter sido escrito no Brasil, o texto deve ser redigido na língua portuguesa e todas as palavras estrangeiras devem ser formatadas utilizando itálico. Como esta regra pode ser mudada pela ABNT ou pela biblioteca, o comando \verb+\outralingua{}+ foi criado. Atualmente, este comando só utiliza o \verb+\emph+ para formatar o texto passado.
    
    Exemplo: \verb+\outralingua{Hello World!}+
    
    \subsection{\textbackslash aspas\{\}}
    As aspas no \LaTeX são geradas de forma diferente dos outros editores de texto e pode ser encontrada em qualquer manual sobre \LaTeX. Apenas para facilitar a inserção de aspas no formato do \LaTeX, foi criado o comando \verb+\aspas{}+ que recebe o texto desejado e o coloca entre aspas.

    Exemplo: \verb+\aspas{Texto entre aspas}+ $\to$ \aspas{Texto entre aspas}
    
    \subsection{\textbackslash marca\{\}}
    É comum precisar que certas células de uma tabela precisem ser destacadas das demais, como em cronogramas, por exemplo. O comando \verb+\marca{}+ foi feito para que a célula de uma tabela ficasse com o fundo cinza. Este é o único comando que utiliza o pacote \verb+xcolor+.

    Exemplo:
    \begin{verbatim}
    \begin{table}[ht]
        \begin{center}
        \begin{tabular}{|c|c|c|}
        \hline
        1 & 2 & 3 \\
        \hline
        \marca{} & & \marca{} \\
        \hline
        a & b & c \\
        \hline
        \end{tabular}
        \end{center}
    \end{table}
    \end{verbatim}
    Resultado: 
    \begin{table}[ht]
        \begin{center}
        \begin{tabular}{|c|c|c|}
        \hline
        1 & 2 & 3 \\
        \hline
        \marca{} & & \marca{} \\
        \hline
        a & b & c \\
        \hline
        \end{tabular}
        \end{center}
    \end{table}
    
    
    \subsection{\textbackslash palavraschave\{\} e \textbackslash keyword\{\}}
    Segundo o modelo da biblioteca da FEI, o resumo e o abstract devem receber no máximo 3 palavras chave. Estes comandos devem ser utilizados dentro dos respectivos ambientes e as palavras devem ser passadas como argumentos.

    Exemplo:
    \begin{verbatim}
    \begin{resumo}
    Aqui deve ser escrito o resumo do trabalho.

    \palavraschave{Resumo. Modelo da FEI. Latex}
    \end{resumo}
    \end{verbatim}
    
    \subsection{\textbackslash apendice\{\}, \textbackslash apendices\{\}, \textbackslash anexo\{\} e \textbackslash anexos\{\}}
    Para os apêndices e anexos foram criados dois comandos separados.

    O modelo da FEI requer que os apêndices e anexos sejam numerados apenas quando existe mais de um no trabalho. Caso exista apenas um anexo ou apêndice, este não leva número sequencial.

    Exemplos: \\
    \verb+\apendice{Único apêndice do trabalho}+ \\
    \verb+\anexo{Único anexo do trabalho}+ \\

    \noindent{}
    \verb+\apendices{Primeiro apêndice}+\\
    \verb+\apendices{Segundo apêndice}+\\

    \noindent{}
    \verb+\anexos{Primeiro anexo}+\\
    \verb+\anexos{Segundo anexo}+\\

    \subsection{\textbackslash bibliografia\{\}}
    A utilização de referências bibliográficas a partir do \texttt{bibtex} depende do comando \verb+\bibliography{}+ que recebe o caminho até o arquivo \texttt{.bib} utilizado. Porém, a adição da página de referências ao sumário e a formatação do título da mesma dependem de outras variáveis que precisam ser definidas durante a produção do texto (o pacote \verb+babel+ substitui o nome da página de referências e este só pode ser mudado após o início do texto).

    Para facilitar, foi criado o comando \verb+\bibliografia{}+ que recebe como parâmetro o caminho para o arquivo \texttt{.bib}. Este comando realiza a formatação necessária e repassa o caminho para o comando \verb+\bibliography{}+ padrão do \LaTeX.

    Exemplo: \verb+\bibliografia{minha_bibliografia.bib}+

